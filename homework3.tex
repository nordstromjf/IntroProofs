\documentclass{article}
\usepackage{amssymb}
\pagestyle{empty}

\setlength{\topmargin}{-.5in}
\setlength{\oddsidemargin}{-.25in}
\setlength{\textwidth}{6in}
\setlength{\textheight}{9in}

\everymath{\displaystyle}

\begin{document}


{\bf Math 220}\hfill{Name:}

{\bf Homework 3: Direct Proof}
\vspace{.3in}






Write all proofs carefully and neatly on your own paper.

\begin{enumerate}
\item  Prove or disprove the following statements using the method of direct proof 
or counter-example.

\begin{enumerate}
\item The difference of any two odd integers is odd.

\smallskip

\item The product of any even integer and any integer is even.

\smallskip

\item The difference of any two even integers is even.

\smallskip

\item The difference of any two odd integers is even.

\smallskip

\item For all integers $n$, if $n$ is prime then $(-1)^n=-1$.

\smallskip

\item For all integers $n$, $n^2-n+11$ is prime.

\smallskip

\item If $m$ and $n$ are perfect squares, then $m+n+2(mn)^{1/2}$ is a perfect square.


\end{enumerate}

\item Prove or disprove the following statements using the method of direct proof 
or counter-example. If the statement is false determine whether a small change would
make it true. If so, make the change and prove the new statement.

\begin{enumerate}
\item The quotient of any two rational numbers is a rational number.

\smallskip

\item Given any two distinct rational numbers $r$ and $s$ with $r<s$, there is a
rational number $x$ such that $r<x<s$.

\smallskip

\item Suppose $a$, $b$, $c$, and $d$ are integers, and $a$ doesn't equal $c$. Suppose
also that $x$ is a real number that satisfies the equation
$${ax+b\over cx+d}=1$$
Then, $x$ is rational.

\end{enumerate}
\item Prove or disprove the following statements using the method of direct proof 
or counter-example.

\begin{enumerate}
\item For all integers $a$, $b$, and $c$, if $a\bigm|b$ and $a\bigm|c$, then
$a\bigm|(b+c)$.

\smallskip

\item The sum of any three consecutive integers is divisible by 3.

\smallskip

\item For all integers $a$, $b$, and $c$, if $a\bigm|b$ then $a\bigm|(bc)$.

\smallskip

\item For all integers $a$, $b$, and $c$, if $a$ is a factor of $c$, then $ab$ is a
factor of $c$.

\smallskip

\item For all integers $a$, $b$, and $c$, if $a\bigm|(b+c)$, then $a\bigm|b$ and
$a\bigm|c$.

\smallskip

\item For all integers $a$, $b$, and $c$, if $a\bigm|(bc)$ then $a\bigm|b$ or
$a\bigm|c$.

\smallskip

\item For all integers $a$ and $b$, if $a\bigm|b$ then $a^2\bigm|(b^2)$.

\smallskip


\end{enumerate}

\item A fast food chain has a contest in which a card with numbers on it is 
given to each customer who makes a purchase. If some of the numbers on the card add
up to 100, then the customer wins \$100. A certain customer receives a card with the
numbers

$$72, 21, 15, 36, 69, 81, 9, 27, 42,\ {\rm and}\ 63$$.

Will the customer win \$100? Prove or disprove your claim.


\end{enumerate}
\end{document}